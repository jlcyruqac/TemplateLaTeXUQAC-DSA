\begin{introduction}

L’introduction présente l’idée principale de la recherche, la problématique, les objectifs et la méthodologie utilisée. Sans entrer dans les détails, l’auteur mentionne les principaux travaux antérieurs ainsi que les aspects originaux de sa recherche.

Le texte contenu dans chacune des sections de l’introduction est écrit à double interligne. De plus, un espacement doit être inséré entre les paragraphes. Le caractère de rédaction est de taille 10 à 12 points pour l’ensemble du texte, sauf en ce qui concerne les citations et les notes de bas de page où le minimum permis est de 8 points. La première ligne de chacun des paragraphes débute avec un retrait.

Tout le corps du mémoire, de l’essai doctoral ou de la thèse est paginé en chiffre arabe. Pour effectuer le changement de pagination, passage du format romain au format arabe, il faut recommencer la pagination. L’introduction commence par le numéro de page 1. Le numéro de page est placé, sans point ni tiret, en bas à droite à 10 mm du bord de page. On rappelle ici que sur toutes les pages portant un titre quelconque (la page de titre, l'avant-propos, la table des matières, un chapitre, la conclusion, etc.), le numéro de la page ne doit pas être visible, mais comptabilisé dans la pagination.

\end{introduction}
