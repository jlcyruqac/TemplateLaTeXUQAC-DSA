\begin{preface}

S’il y a lieu, l’avant-propos permet à l’auteur de faire état des raisons qui l’ont conduit à choisir un sujet. L’avant-propos présente le but ainsi que l’ampleur et les limites de la recherche. Ce dernier ne doit pas être confondu avec l’introduction.

Le texte de l’avant-propos est écrit à simple interligne et la pagination est en chiffres romains, en petites capitales. Le caractère de rédaction est de taille 10 à 12 points pour l’ensemble du texte, sauf en ce qui concerne les citations et les notes de bas de page où le minimum permis est de 8 points. La première ligne de chacun des paragraphes débute avec un retrait.

Sur les pages portant un titre quelconque (la page de titre, l’avant-propos, la table des matières, un chapitre, la conclusion, etc.), le numéro de la page ne doit pas être visible, mais comptabilisé dans la pagination.

La page de titre, le résumé, la table des matières, la liste des tableaux, la liste des figures, la liste des sigles, la liste des abréviations, la dédicace, les remerciements et l’avant-propos sont paginés en chiffres romains, en petites capitales. Le numéro de page doit être placé, sans point ni tiret, en bas à droite à 10 mm du bord de page.

\end{preface}
