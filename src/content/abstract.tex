\begin{abstract}

Un résumé (présentation des idées maîtresses et des conclusions de l’ouvrage) doit être placé au début. L’étudiant peut ajouter un résumé en anglais qui sera intitulé « Abstract ». L'abstract doit être une traduction anglaise fidèle et grammaticalement correcte du résumé en français, dans le même format que le résumé. 

Exceptionnellement, le doyen des études peut autoriser l’étudiant à rédiger son mémoire ou sa thèse dans une autre langue que le français. Dans ce cas, l’étudiant doit rédiger un résumé en français.

Toutes les marges de ce document ont été définies à 25 mm à l’exception de celles de gauche qui sont fixées à 35 mm pour permettre la reliure. Le résumé doit être écrit à simple interligne. Le caractère de rédaction doit être de taille 10 à 12 points pour l’ensemble du texte, sauf en ce qui concerne les citations et les notes de bas de page où le minimum permis est de 8 points. Seule la mise en page recto est autorisée et la dimension des pages doit être uniforme (21,5 cm x 28 cm). La première ligne de chacun des paragraphes débute avec un retrait.

Sur les pages portant un titre quelconque (la page de titre, l’avant-propos, la table des matières, un chapitre, la conclusion, etc.), le numéro de la page ne doit pas être visible, mais comptabilisé dans la pagination.

La page de titre, le résumé, la table des matières, la liste des tableaux, la liste des figures, la liste des sigles, la liste des abréviations, la dédicace, les remerciements et l’avant-propos sont paginés en chiffres romains, en petites capitales. Le numéro de page doit être placé, sans point ni tiret, en bas à droite à 10 mm du bord de page.

Enfin, aux fins d’archivage, il est préférable que le résumé ne dépasse pas 5 000 caractères, incluant les espaces.

Pour construire la table des matières qui suit, dans l’onglet Références, sélectionner le texte du titre ou du sous-titre dans le document et «ajouter le texte» au niveau 1 pour les titres et au niveau 2 pour les sous-titres. Pour mettre à jour la table des matières, cliquer sur la table des matières et sélectionner «mettre à jour la table». Choisissez «mettre à jour les numéros de pages uniquement» ou « mettre à jour la table».

Les exigences de rédaction et de présentation d’un mémoire ou d’une thèse pour les programmes d’études offerts à l’UQAC, mais qui relèvent d’une autre université (extension, association ou programme réseau) peuvent différer. Pour connaître ces exigences, l’étudiant doit communiquer avec le secrétariat de son programme.

\end{abstract}
