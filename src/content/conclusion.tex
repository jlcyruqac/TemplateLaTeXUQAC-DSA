\begin{conclusion}

La conclusion permet de mettre en évidence la portée de l’étude. C’est dans cette section que l’on y fait état des limites de la recherche et, le cas échéant,  des pistes nouvelles pour de futures recherches. La conclusion ne doit pas présenter de nouvelles idées, de nouveaux résultats ou de nouvelles interprétations. Elle doit être rédigée de façon à faire ressortir la cohérence de la démarche.

Comme pour les chapitres, le titre « conclusion » s’écrit en lettres majuscules et non souligné. Le texte contenu dans cette section doit être écrit à double interligne. De plus, un espacement doit être inséré entre les paragraphes. Le caractère de rédaction est de taille 10 à 12 points pour l’ensemble du texte. La première ligne de chacun des paragraphes débute avec un retrait. De plus, un espacement doit être inséré entre les paragraphes.

La pagination de cette section doit suivre celle de la partie précédente en chiffre arabe. Le numéro de page est placé, sans point ni tiret, en bas à droite à 10 mm du bord de page. On rappelle ici que sur toutes les pages portant un titre quelconque (la page de titre, l'avant-propos, la table des matières, un chapitre, la conclusion, etc.), le numéro de la page ne doit pas être visible, mais comptabilisé dans la pagination.

\end{conclusion}
