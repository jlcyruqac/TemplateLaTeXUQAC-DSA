%%%%%%%%%%%%%%%%%%%%%%%%%%%%%%%%%%%%%%%%%%%%%%%%%%%%%%%%%%%%%%%%%%%%%%%%%%%%%
%%
%% Classes disponibles :
%%	- times : utilisation de la police Times par défaut.
%%	- these, sujthese, projthese, memoire, essai, rapport : pour
%%		générer la page de garde en fonction du type document.
%%	- french, english : selon la langue dans laquelle doit
%%		être rédigé le document.
%%  - ieee, apa : style des références bibliographiques
%%  - twoside : alternance des marges pour impression recto/verso
%%    vous devez décommenter la ligne 22 si vous utilisez cette classe
%%
%%%%%%%%%%%%%%%%%%%%%%%%%%%%%%%%%%%%%%%%%%%%%%%%%%%%%%%%%%%%%%%%%%%%%%%%%%%%%%
\documentclass[12pt,times,these,french,apa]{uqac}

%%%%%%%%%%%%%%%%%%%%%%%%%%%%%%%%%%%%%%%%%%%%%%%%%%%%%%%
%%
%% Décommenter la ligne suivante avec l'utilisation de
%% la classe "twoside"
%%
%%%%%%%%%%%%%%%%%%%%%%%%%%%%%%%%%%%%%%%%%%%%%%%%%%%%%%%
% \raggedbottom

%%%%%%%%%%%%%%%%%%%%%%%%%%%%%%%%%%%%%
%%
%% Préciser l'emplacement du fichier
%% contenant les acronymes.
%%
%%%%%%%%%%%%%%%%%%%%%%%%%%%%%%%%%%%%%
\acrolistpath{assets/acro}

%%%%%%%%%%%%%%%%%%%%%%%%%%%%%%%%%%%%%%%%%%%%%%%%%%%%%%%%
%%
%% Toutes les images utilisées doivent se trouver dans
%% le répertoire "assets/figure".
%%
%%%%%%%%%%%%%%%%%%%%%%%%%%%%%%%%%%%%%%%%%%%%%%%%%%%%%%%%
\graphicspath{{assets/figures}}

%%%%%%%%%%%%%%%%%%%%%%%%%%%%%%%%%%%%
%%
%% Packages utilisateurs ci-dessous:
%%
%%%%%%%%%%%%%%%%%%%%%%%%%%%%%%%%%%%%
%\usepackage{color, xcolor, soulutf8}
\usepackage{color, xcolor}
\usepackage{tikz}
\usetikzlibrary{babel}
\usepackage{chemfig}
\usepackage{chemnum}
\usepackage[europeanresistors,americaninductors]{circuitikz}
\usepackage{tikz-feynman}
\usepackage{hyperref}
\hypersetup{
  colorlinks=true,
  linkcolor=black,
  urlcolor=[rgb]{0,0.501,0.674},
  breaklinks=true,
  citecolor=[rgb]{0,0.501,0.674},
  bookmarks=true
}

%%%%%%%%%%%%%%%%%%%%%
%%
%% Césures manuelles
%%
%%%%%%%%%%%%%%%%%%%%%
% \hyphenation{con-nection con-nue con-cep-tion}

%%%%%%%%%%%%%%%%%%%%%%%%%%%%%%%%
%%
%% Gestion des lignes orphelines
%%
%%%%%%%%%%%%%%%%%%%%%%%%%%%%%%%%
\widowpenalty10000
\clubpenalty10000

\begin{document}

%%%%%%%%%%%%%%%%%%%%%%%%%%%%%
%%
%% Titre, auteur et programme
%%
%%%%%%%%%%%%%%%%%%%%%%%%%%%%%

\title{Titre de la thèse du mémoire ou de l'essai}
\subtitle{Sous-titre s'il y a lieu}
\author{Prénom Nom}
\director{directeur et co-directeurs}
\programme{nom du programme (-s'il y a lieu, profil ou concentation)}
\soutenancedate{jour mois année}
\jury{Prénom Nom, fonction, affiliation, Président du Jury\\Prénom Nom, fonction, affiliation, Membre externe ou rapporteur\\Prénom Nom, fonction, affiliation, Membre interne\\Prénom Nom, fonction, affiliation, Membre interne}

%%%%%%%%%%%%%%%%%%%%%%%%%%%%%%%%%%%%%
%%
%% S’il y a lieu, indiquer le profil
%% ou la concentration du programme
%%
%%%%%%%%%%%%%%%%%%%%%%%%%%%%%%%%%%%%%
%\concentration{profil recherche}

%%%%%%%%%%%%%%%%%%%%%%%%%%%%%%%%%%%%%%%%%%%%
%%
%% Par défaut l'année courante est utilisée.
%% Pour spécifier une autre année :
%%
%%%%%%%%%%%%%%%%%%%%%%%%%%%%%%%%%%%%%%%%%%%%
%\degreeyear{2018}

\maketitle

%%%%%%%%%%%%%%%%%%%%%
%%
%% Page préliminaires
%%
%%%%%%%%%%%%%%%%%%%%%
\opening
% \pagestyle{empty}

\begin{abstract}

Un résumé (présentation des idées maîtresses et des conclusions de l’ouvrage) doit être placé au début. L’étudiant peut ajouter un résumé en anglais qui sera intitulé « Abstract ». L'abstract doit être une traduction anglaise fidèle et grammaticalement correcte du résumé en français, dans le même format que le résumé. 

Exceptionnellement, le doyen des études peut autoriser l’étudiant à rédiger son mémoire ou sa thèse dans une autre langue que le français. Dans ce cas, l’étudiant doit rédiger un résumé en français.

Toutes les marges de ce document ont été définies à 25 mm à l’exception de celles de gauche qui sont fixées à 35 mm pour permettre la reliure. Le résumé doit être écrit à simple interligne. Le caractère de rédaction doit être de taille 10 à 12 points pour l’ensemble du texte, sauf en ce qui concerne les citations et les notes de bas de page où le minimum permis est de 8 points. Seule la mise en page recto est autorisée et la dimension des pages doit être uniforme (21,5 cm x 28 cm). La première ligne de chacun des paragraphes débute avec un retrait.

Sur les pages portant un titre quelconque (la page de titre, l’avant-propos, la table des matières, un chapitre, la conclusion, etc.), le numéro de la page ne doit pas être visible, mais comptabilisé dans la pagination.

La page de titre, le résumé, la table des matières, la liste des tableaux, la liste des figures, la liste des sigles, la liste des abréviations, la dédicace, les remerciements et l’avant-propos sont paginés en chiffres romains, en petites capitales. Le numéro de page doit être placé, sans point ni tiret, en bas à droite à 10 mm du bord de page.

Enfin, aux fins d’archivage, il est préférable que le résumé ne dépasse pas 5 000 caractères, incluant les espaces.

Pour construire la table des matières qui suit, dans l’onglet Références, sélectionner le texte du titre ou du sous-titre dans le document et «ajouter le texte» au niveau 1 pour les titres et au niveau 2 pour les sous-titres. Pour mettre à jour la table des matières, cliquer sur la table des matières et sélectionner «mettre à jour la table». Choisissez «mettre à jour les numéros de pages uniquement» ou « mettre à jour la table».

Les exigences de rédaction et de présentation d’un mémoire ou d’une thèse pour les programmes d’études offerts à l’UQAC, mais qui relèvent d’une autre université (extension, association ou programme réseau) peuvent différer. Pour connaître ces exigences, l’étudiant doit communiquer avec le secrétariat de son programme.

\end{abstract}


%%%%%%%%%%%%%%%%%%%%%%%%%%%%%%%%%%%%%%%%%%%%%%%%%%%%%%%%%%
%%
%% Si vous écrivez un document en anglais,
%% il sera nécessaire de fournir un résumé en Français :
%%
%%%%%%%%%%%%%%%%%%%%%%%%%%%%%%%%%%%%%%%%%%%%%%%%%%%%%%%%%%
%\include{content/resume}

\tableofcontents
\cleardoublepage
\listoftables
\listoffigures
\listofacro

\begin{dedic}

S’il y a lieu, l’auteur peut rendre hommage à une ou plusieurs personnes.

Le texte de la dédicace est écrit à simple interligne et la pagination est en chiffres romains, en petites capitales. Le caractère de rédaction doit être de taille 10 à 12 points pour l’ensemble du texte, sauf en ce qui concerne les citations et les notes de bas de page où le minimum permis est de 8 points. La première ligne de chacun des paragraphes débute avec un retrait.

Sur les pages portant un titre quelconque (la page de titre, l’avant-propos, la table des matières, un chapitre, la conclusion, etc.), le numéro de la page ne doit pas être visible, mais comptabilisé dans la pagination. 

La page de titre, le résumé, la table des matières, la liste des tableaux, la liste des figures, la liste des sigles, la liste des abréviations, la dédicace, les remerciements et l’avant-propos sont paginés en chiffres romains, en petites capitales. Le numéro de page doit être placé, sans point ni tiret, en bas à droite à 10 mm du bord de page.

\end{dedic}

\begin{ack}

S’il y a lieu, les remerciements permettent d’exprimer la reconnaissance de l’auteur envers des personnes ou des organismes.

Le texte des remerciements est écrit à simple interligne et la pagination est en chiffres romains, en petites capitales. Le caractère de rédaction doit être de taille 10 à 12 points pour l’ensemble du texte, sauf en ce qui concerne les citations et les notes de bas de page où le minimum permis est de 8 points. La première ligne de chacun des paragraphes débute avec un retrait.

Sur les pages portant un titre quelconque (la page de titre, l’avant-propos, la table des matières, un chapitre, la conclusion, etc.), le numéro de la page ne doit pas être visible, mais comptabilisé dans la pagination.

La page de titre, le résumé, la table des matières, la liste des tableaux, la liste des figures, la liste des sigles, la liste des abréviations, la dédicace, les remerciements et l’avant-propos sont paginés en chiffres romains, en petites capitales. Le numéro de page doit être placé, sans point ni tiret, en bas à droite à 10 mm du bord de page.

\end{ack}

\begin{preface}

S’il y a lieu, l’avant-propos permet à l’auteur de faire état des raisons qui l’ont conduit à choisir un sujet. L’avant-propos présente le but ainsi que l’ampleur et les limites de la recherche. Ce dernier ne doit pas être confondu avec l’introduction.

Le texte de l’avant-propos est écrit à simple interligne et la pagination est en chiffres romains, en petites capitales. Le caractère de rédaction est de taille 10 à 12 points pour l’ensemble du texte, sauf en ce qui concerne les citations et les notes de bas de page où le minimum permis est de 8 points. La première ligne de chacun des paragraphes débute avec un retrait.

Sur les pages portant un titre quelconque (la page de titre, l’avant-propos, la table des matières, un chapitre, la conclusion, etc.), le numéro de la page ne doit pas être visible, mais comptabilisé dans la pagination.

La page de titre, le résumé, la table des matières, la liste des tableaux, la liste des figures, la liste des sigles, la liste des abréviations, la dédicace, les remerciements et l’avant-propos sont paginés en chiffres romains, en petites capitales. Le numéro de page doit être placé, sans point ni tiret, en bas à droite à 10 mm du bord de page.

\end{preface}


%%%%%%%%%%%%%%%%%%%%%
%%
%% Document principal
%%
%%%%%%%%%%%%%%%%%%%%%
% \pagestyle{fancy}
\maincontent

\begin{introduction}

L’introduction présente l’idée principale de la recherche, la problématique, les objectifs et la méthodologie utilisée. Sans entrer dans les détails, l’auteur mentionne les principaux travaux antérieurs ainsi que les aspects originaux de sa recherche.

Le texte contenu dans chacune des sections de l’introduction est écrit à double interligne. De plus, un espacement doit être inséré entre les paragraphes. Le caractère de rédaction est de taille 10 à 12 points pour l’ensemble du texte, sauf en ce qui concerne les citations et les notes de bas de page où le minimum permis est de 8 points. La première ligne de chacun des paragraphes débute avec un retrait.

Tout le corps du mémoire, de l’essai doctoral ou de la thèse est paginé en chiffre arabe. Pour effectuer le changement de pagination, passage du format romain au format arabe, il faut recommencer la pagination. L’introduction commence par le numéro de page 1. Le numéro de page est placé, sans point ni tiret, en bas à droite à 10 mm du bord de page. On rappelle ici que sur toutes les pages portant un titre quelconque (la page de titre, l'avant-propos, la table des matières, un chapitre, la conclusion, etc.), le numéro de la page ne doit pas être visible, mais comptabilisé dans la pagination.

\end{introduction}

\chapter{premier chapitre}
\label{chap:chap_1}

\section{exemples de base}
\label{sec:sec_examples_chap_1}

Chaque chapitre est identifié en lettres majuscules et non souligné. Le titre du chapitre est centré et se place sous la numérotation du chapitre, en lettres majuscules, non souligné, sans point final. Si le titre comporte plus d'une ligne, ce dernier est écrit à simple interligne. Les divers sous-titres doivent s'identifier facilement par une présentation uniforme dans l'ensemble du document.

Le texte contenu dans chacune de ces sections doit être écrit à double interligne. De plus, un espacement doit être inséré entre les paragraphes. Le caractère de rédaction est de taille 10 à 12 points pour l’ensemble du texte, sauf en ce qui concerne les citations et les notes de bas de page où le minimum permis est de 8 points. La première ligne de chacun des paragraphes débute avec un retrait.

Tout le corps du mémoire, de l’essai doctoral ou de la thèse est paginé en chiffre arabe. Le numéro de page est placé, sans point ni tiret, en bas à droite à 10 mm du bord de page. On rappelle ici que sur toutes les pages portant un titre quelconque (la page de titre, l'avant-propos, la table des matières, un chapitre, la conclusion, etc.), le numéro de la page ne doit pas être visible, mais comptabilisé dans la pagination.

\subsection{listes}
\label{subsec:subsec_list}

\begin{enumerate}
  \item {\bf Premier point (gras) ;}
  \item {\em Second point (italique) ;}
  \item {\Large Troisième point (gros) ;}
      \begin{enumerate}
          \item {\small Premier sous-point en petit}
          \item {\tiny Second sous-point (petit)}
          \item {\Huge Troisième sous-point (très gros)}
      \end{enumerate}
  \item[$\bullet$] {\sf Point avec une puce (sans serif)}
  \item[$\circ$] {\sc Point avec un autre style de puce (petites lettres capitales)}
\end{enumerate}

\subsection{citations}

Ceci est un premier exemple de citation standard \cite{RN3}. Vous pouvez ensuite forcer l'utilisation des citation entre parenthèse pour le format APA-Provost tel que, \citep{RN4}. Enfin, vous pouvez également citer les auteurs \citet{RN5} de cette manière, ou ne citer que l'année de publication en  \citeyear{RN5}.

En ce qui concerne les citations plus longues, ces dernières sont détachées du texte, entre guillemets, à simple interligne ainsi qu’en retrait d’au moins 10 mm à droite et à gauche. Il faut donc procéder ainsi :
\begin{displayquote}
Analyser signifie décomposer un phénomène de manière à en distinguer les éléments constitutifs. Cette division d’un phénomène global en plus petits éléments est effectuée dans le but de reconnaître ou d’expliquer les rapports qui lient ces éléments entre eux afin de mieux comprendre le phénomène dans sa globalité. L’analyse prépare la synthèse qui, elle, correspond à l’opération visant à lier les éléments identifiés pour former un nouvel ensemble cohérent. \citep[p.104]{RN4}
\end{displayquote}

\subsection{liens}

Si vous utilisez \href{https://code.visualstudio.com/}{Visual Studio Code} pour composer votre document \LaTeX{}, vous pouvez utiliser l'extension \textit{vscode-ltex} disponible \textit{via} le lien suivant : \url{https://github.com/valentjn/vscode-ltex} pour corriger vos erreurs d'orthographe et de grammaire.

\subsection{notes de bas de page}

Vous pouvez utiliser les notes de bas de page pour inclure un lien\footnote{\url{https://www.uqac.ca}} en rapport avec votre texte, ou pour donner plus de précisions\footnote{Cette note de bas de page propose plus d'informations}.

\subsection{acronymes}

Exemple de définition d'un acronyme de trois lettres : \ac{TLA}. Puis utilisation de cet acronyme en version courte \acs{TLA}. Vous pouvez également mettre au pluriel l'acronyme long tel que, \aclp{TLA} ainsi que l'acronyme court : \acsp{TLA}.

\subsection{figures}

\begin{figure}[H]
 \centering
 \includegraphics[width=7cm]{lenna.png}
 \caption{Titre de la figure qui doit normalement se tenir sur deux lignes dans la liste des figures.}
 \label{fig:figure_long}
\end{figure}

Vous pouvez ensuite mentionner la Figure \ref{fig:figure_long} dans votre texte.

\begin{figure}[H]
  \centering
  \includegraphics[width=7cm]{lenna.png}
  \caption[Titre de la figure sans citation]{Titre de la figure qui comporte une citation de \citet{RN6}.}
  \label{fig:figure_cite}
 \end{figure}

 Vous pouvez aussi référencer la Figure \ref{fig:figure_cite} directement dans votre texte.

\subsection{tableaux}

\begin{table}[H]
  \begin{center}
    \caption{Titre du tableau qui doit normalement se tenir sur deux lignes dans la liste des tableaux.}
    \label{tab:tab_1}
    \begin{tabular}{l|c|r}
      \textbf{Value 1} & \textbf{Value 2} & \textbf{Value 3}\\
      $\alpha$ & $\beta$ & $\gamma$ \\
      \hline
      1 & 1110.1 & a\\
      2 & 10.1 & b\\
      3 & 23.113231 & c\\
    \end{tabular}
  \end{center}
\end{table}

Vous pouvez ensuite mentionner le Tableau \ref{tab:tab_1} dans votre texte.

\include{content/chapitre2}
\chapter{troisième chapitre}

\section{Informatique}

Vous pouvez inclure du code de la manière suivante :

%\begin{minted}[frame=single,framesep=2mm,baselinestretch=1,fontsize=\footnotesize,linenos]{cpp}
%{
%  #include <iostream>
%
%  int main() {
%      std::cout << "Hello World!";
%      return 0;
%  }
%}
%\end{minted}

\section{Chimie}

Vous pouvez inclure des formules chimiques de la manière suivante :

\begin{figure}[H]
  \centering
  \chemfig{-[:30](=[:90]O)-[:-30]OH}
  \caption{La formule de l'Éthanol.}
  \label{fig:ethanol}
\end{figure}

\noindent La formule de l'Éthanol est représentée en Figure \ref{fig:ethanol}.

\section{Ingénierie}

Vous pouvez inclure des schémas électriques de la manière suivante :

\begin{figure}[H]
  \centering
  \begin{circuitikz}[american voltages]
    \draw
      (0,0) to [short, *-] (6,0)
      to [V, l_=$\mathrm{j}{\omega}_m \underline{\psi}^s_R$] (6,2)
      to [R, l_=$R_R$] (6,4)
      to [short, i_=$\underline{i}^s_R$] (5,4)

      (0,0) to [open, v^>=$\underline{u}^s_s$] (0,4)
      to [short, *- ,i=$\underline{i}^s_s$] (1,4)
      to [R, l=$R_s$] (3,4)
      to [L, l=$L_{\sigma}$] (5,4)
      to [short, i_=$\underline{i}^s_M$] (5,3)
      to [L, l_=$L_M$] (5,0);
  \end{circuitikz}
  \caption{Exemple d'un schéma électriques.}
  \label{fig:elec}
\end{figure}

\noindent La Figure \ref{fig:elec} montre un exemple de schéma électriques.

\section{Physique}

Vous pouvez inclure des diagrammes de Feynman de la manière suivante :

\begin{figure}[H]
  \centering
  \feynmandiagram [small, horizontal=a to t1] {
    a [particle=\(\pi^{0}\)] -- [scalar] t1 -- t2 -- t3 -- t1,
    t2 -- [photon] p1 [particle=\(\gamma\)],
    t3 -- [photon] p2 [particle=\(\gamma\)],
  };
  \caption{Exemple d'un diagramme de Feynman.}
  \label{fig:feynman}
\end{figure}

\noindent La Figure \ref{fig:feynman} montre l'exemple d'un diagramme de Feynman.

\begin{conclusion}

La conclusion permet de mettre en évidence la portée de l’étude. C’est dans cette section que l’on y fait état des limites de la recherche et, le cas échéant,  des pistes nouvelles pour de futures recherches. La conclusion ne doit pas présenter de nouvelles idées, de nouveaux résultats ou de nouvelles interprétations. Elle doit être rédigée de façon à faire ressortir la cohérence de la démarche.

Comme pour les chapitres, le titre « conclusion » s’écrit en lettres majuscules et non souligné. Le texte contenu dans cette section doit être écrit à double interligne. De plus, un espacement doit être inséré entre les paragraphes. Le caractère de rédaction est de taille 10 à 12 points pour l’ensemble du texte. La première ligne de chacun des paragraphes débute avec un retrait. De plus, un espacement doit être inséré entre les paragraphes.

La pagination de cette section doit suivre celle de la partie précédente en chiffre arabe. Le numéro de page est placé, sans point ni tiret, en bas à droite à 10 mm du bord de page. On rappelle ici que sur toutes les pages portant un titre quelconque (la page de titre, l'avant-propos, la table des matières, un chapitre, la conclusion, etc.), le numéro de la page ne doit pas être visible, mais comptabilisé dans la pagination.

\end{conclusion}


%%%%%%%%$%%%%
%%
%% Références
%%
%%%%%%%%%%%%%

%%%%%%%%%%%%%%%%%%%%%%%%%%%%%%%%%%%%%%%%%%%%%%%%%%%%%%%%%%
%%
%% Le fichier BibTeX contenant les références
%% doit se trouver dans le répertoire "assets/references".
%%
%%%%%%%%%%%%%%%%%%%%%%%%%%%%%%%%%%%%%%%%%%%%%%%%%%%%%%%%%%
\bibliography{assets/references}

%%%%%%%%%%%
%%
%% Annexes
%%
%%%%%%%%%%%
\appendix

\include{content/annexe_a}

\end{document}
